\section*{Abstract}
\setcounter{page}{2}

\subsection*{Background}
% The abstract or summary should contain an outline of the work carried out and any significant results achieved. 
% \begin{itemize}
 % \item Is abstract between 200-300 words?
 % \item Does the abstract describe the nature of the work and the results obtained?
 % \item The abstract should contain key-words.
% \end{itemize}
% 
% The Abstract provides the reader with a summary of the contents of the dissertation. It should therefore be brief but contain sufficient detail, telling the reader the motivation for the work; project objectives; techniques employed; main results and conclusions. Abstracts should not normally exceed a page and should be self-contained.
% 
% The Abstract is the "gateway" to the contents of the dissertation, and therefore it is important that the Abstract gives the reader a good initial impression. Try to write Abstract with a "punchy" style.
% 
% (Write the Abstract last. The dissertation will be easier to summarise once all the bits are in place.)



Protein are annotated with domain and architecture information. However exploring the architecture landscape around a protein is currently challenging and there aren't any tools that allow bioinformaticians to visually browse and discover related proteins based on their domains and architectures. 

 

\subsection*{Results}
% Results go here
This paper presents DomVizApp and how it can be used to explore protein architectures. It is a desktop application which uses network graphs to visualise architecture similarities. The introduction describes the why domains and architectures are important, especially for discovering new drug targets. We then describe the data that was used along with the implementation of DomVizApp. We present the results of our data analysis and **"a case study is used to show how DomVizApp can help"** to explore architectures and discover new druggable proteins.
\subsection*{Conclusions}
% Conclusions go here
\thispagestyle{empty}
%\end{abstract}

